% Options for packages loaded elsewhere
\PassOptionsToPackage{unicode}{hyperref}
\PassOptionsToPackage{hyphens}{url}
\PassOptionsToPackage{dvipsnames,svgnames,x11names}{xcolor}
%
\documentclass[
  12pt,
  letterpaper,
]{article}

\usepackage{xcolor}
\usepackage[margin=1in]{geometry}
\usepackage{amsmath,amssymb}
\setcounter{secnumdepth}{-\maxdimen} % remove section numbering
\usepackage{iftex}
\ifPDFTeX
  \usepackage[T1]{fontenc}
  \usepackage[utf8]{inputenc}
  \usepackage{textcomp} % provide euro and other symbols
\else % if luatex or xetex
  \usepackage{unicode-math}
  \defaultfontfeatures{Scale=MatchLowercase}
  \defaultfontfeatures[\rmfamily]{Ligatures=TeX,Scale=1}
\fi
\usepackage{lmodern}
\ifPDFTeX\else
  % xetex/luatex font selection
\fi

% Use upquote if available, for straight quotes in verbatim environments
\IfFileExists{upquote.sty}{\usepackage{upquote}}{}
\IfFileExists{microtype.sty}{%
  \usepackage[]{microtype}
  \UseMicrotypeSet[protrusion]{basicmath} % disable protrusion for tt fonts
}{}

% Line numbering
\usepackage{lineno}
\linenumbers

% Page headers and footers
\usepackage{fancyhdr}
\usepackage{lastpage}
\pagestyle{fancy}
\fancyhf{} % clear all header and footer fields
\fancyhead[L]{\textit{Science Advances}}
\fancyhead[R]{\textit{Manuscript Template}}
\fancyfoot[R]{Page \thepage\ of \pageref{LastPage}}
\renewcommand{\headrulewidth}{0pt}

\usepackage{setspace}
\makeatletter
\@ifundefined{KOMAClassName}{% if non-KOMA class
  \IfFileExists{parskip.sty}{%
    \usepackage{parskip}
  }{% else
    \setlength{\parindent}{0pt}
    \setlength{\parskip}{6pt plus 2pt minus 1pt}}
}{% if KOMA class
  \KOMAoptions{parskip=half}}
\makeatother

% Make \paragraph and \subparagraph free-standing
\makeatletter
\ifx\paragraph\undefined\else
  \let\oldparagraph\paragraph
  \renewcommand{\paragraph}{
    \@ifstar
      \xxxParagraphStar
      \xxxParagraphNoStar
  }
  \newcommand{\xxxParagraphStar}[1]{\oldparagraph*{#1}\mbox{}}
  \newcommand{\xxxParagraphNoStar}[1]{\oldparagraph{#1}\mbox{}}
\fi
\ifx\subparagraph\undefined\else
  \let\oldsubparagraph\subparagraph
  \renewcommand{\subparagraph}{
    \@ifstar
      \xxxSubParagraphStar
      \xxxSubParagraphNoStar
  }
  \newcommand{\xxxSubParagraphStar}[1]{\oldsubparagraph*{#1}\mbox{}}
  \newcommand{\xxxSubParagraphNoStar}[1]{\oldsubparagraph{#1}\mbox{}}
\fi
\makeatother

\usepackage{longtable,booktabs,array}
\usepackage{calc} % for calculating minipage widths
% Correct order of tables after \paragraph or \subparagraph
\usepackage{etoolbox}
\makeatletter
\patchcmd\longtable{\par}{\if@noskipsec\mbox{}\fi\par}{}{}
\makeatother
% Allow footnotes in longtable head/foot
\IfFileExists{footnotehyper.sty}{\usepackage{footnotehyper}}{\usepackage{footnote}}
\makesavenoteenv{longtable}
\usepackage{graphicx}
\makeatletter
\newsavebox\pandoc@box
\newcommand*\pandocbounded[1]{% scales image to fit in text height/width
  \sbox\pandoc@box{#1}%
  \Gscale@div\@tempa{\textheight}{\dimexpr\ht\pandoc@box+\dp\pandoc@box\relax}%
  \Gscale@div\@tempb{\linewidth}{\wd\pandoc@box}%
  \ifdim\@tempb\p@<\@tempa\p@\let\@tempa\@tempb\fi% select the smaller of both
  \ifdim\@tempa\p@<\p@\scalebox{\@tempa}{\usebox\pandoc@box}%
  \else\usebox{\pandoc@box}%
  \fi%
}
% Set default figure placement to htbp
\def\fps@figure{htbp}
\makeatother

\setlength{\emergencystretch}{3em} % prevent overfull lines
\providecommand{\tightlist}{%
  \setlength{\itemsep}{0pt}\setlength{\parskip}{0pt}}

% pandoc-citeproc

\usepackage{hyperref}
\hypersetup{
  pdftitle={Concise, descriptive title in sentence case},
  pdfauthor={Jane Q. Researcher; Taylor P. Collaborator},
  pdfkeywords={keyword one, keyword two, keyword three},
  colorlinks=true,
  linkcolor={blue},
  filecolor={Maroon},
  citecolor={Blue},
  urlcolor={Blue},
  pdfcreator={LaTeX via pandoc}}

\urlstyle{same} % disable monospaced font for URLs

% Title and author information
\title{Concise, descriptive title in sentence case}
\newcommand{\shorttitle}{Short title (max 50 characters)}

% Custom author formatting with affiliations
\makeatletter
\def\@maketitle{%
  \newpage
  \null
  \vskip 1em%
  \begin{flushleft}%
  \let \footnote \thanks
    {\LARGE \textbf{\@title} \par}%
    \vskip 1em%
Jane Q. Researcher, Taylor P. Collaborator    \par
    \vskip 0.5em%
  \end{flushleft}%
  \par
  \vskip 1em}
\makeatother

\date{}

\begin{document}

\maketitle

\begin{abstract}
\noindent
Provide a single paragraph (\textless160 words) accessible to a broad
audience. Avoid citations and include a brief problem statement, key
approach, primary findings, and a short concluding sentence.
\end{abstract}

\noindent
\textbf{One-sentence summary}

\noindent
Single-sentence summary under 130 characters.

\vskip 1em

\noindent
\textbf{Keywords:} keyword one, keyword two, keyword three

\vskip 1em

\setstretch{1.2}

\subsubsection{One-sentence summary}\label{one-sentence-summary}

Single-sentence summary under 130 characters.

\subsubsection{Author notes}\label{author-notes}

\begin{itemize}
\tightlist
\item
  Asterisk (*) marks the corresponding author and should match the
  e-mail in the author block.
\item
  Use symbols (†, ‡, §, \textbar\textbar, ¶, \#, ††, ‡‡) for notes such
  as equal contribution or present address.
\end{itemize}

\subsection{Abstract}\label{abstract}

Replace this paragraph with your abstract. Keep it under 160 words and
avoid citations.

\subsection{Introduction}\label{introduction}

Outline the background and the question or problem being addressed. Keep
the narrative accessible to a multidisciplinary reader.

\subsection{Results}\label{results}

Describe the core findings in logical order. Cite figures and tables
inline (for example, ``Fig. 1'' or ``Table 2'').

\subsection{Discussion}\label{discussion}

Interpret the results, highlight limitations, and connect to prior work
without restating the Results section.

\subsection{Materials and Methods}\label{materials-and-methods}

Provide enough detail for reproducibility. Include statistical tests,
sample sizes, number of replicates, and any preregistration identifiers
when applicable.

\subsection{References}\label{references}

Citations follow Science Advances style via the CSL file. Ensure entries
appear in the order cited.

\subsection{Acknowledgments}\label{acknowledgments}

List any individuals who contributed but are not authors.

\subsection{Funding}\label{funding}

Specify funding sources and grant numbers.

\subsection{Author contributions}\label{author-contributions}

Describe contributions using CRediT-style roles (for example,
``Conceptualization: JR, TC; Investigation: JR; Writing -- original
draft: JR; Writing -- review \& editing: JR, TC'').

\subsection{Competing interests}\label{competing-interests}

Disclose any financial or personal relationships that could influence
the work. State ``The authors declare that they have no competing
interests'' if none exist.

\subsection{Data and materials
availability}\label{data-and-materials-availability}

Detail where data, code, and materials can be accessed (with DOIs or
repository links) and any restrictions.

\subsection{Supplementary Materials}\label{supplementary-materials}

List the sections and figures included in the companion Supplementary
Materials PDF. Example:

\begin{itemize}
\tightlist
\item
  Materials and Methods
\item
  Fig. S1 to S5
\item
  Table S1 to S3
\item
  References (61--75)
\end{itemize}

\end{document}
